% Options for packages loaded elsewhere
% Options for packages loaded elsewhere
\PassOptionsToPackage{unicode}{hyperref}
\PassOptionsToPackage{hyphens}{url}
\PassOptionsToPackage{dvipsnames,svgnames,x11names}{xcolor}
%
\documentclass[
  letterpaper,
  DIV=11,
  numbers=noendperiod]{scrartcl}
\usepackage{xcolor}
\usepackage{amsmath,amssymb}
\setcounter{secnumdepth}{-\maxdimen} % remove section numbering
\usepackage{iftex}
\ifPDFTeX
  \usepackage[T1]{fontenc}
  \usepackage[utf8]{inputenc}
  \usepackage{textcomp} % provide euro and other symbols
\else % if luatex or xetex
  \usepackage{unicode-math} % this also loads fontspec
  \defaultfontfeatures{Scale=MatchLowercase}
  \defaultfontfeatures[\rmfamily]{Ligatures=TeX,Scale=1}
\fi
\usepackage{lmodern}
\ifPDFTeX\else
  % xetex/luatex font selection
\fi
% Use upquote if available, for straight quotes in verbatim environments
\IfFileExists{upquote.sty}{\usepackage{upquote}}{}
\IfFileExists{microtype.sty}{% use microtype if available
  \usepackage[]{microtype}
  \UseMicrotypeSet[protrusion]{basicmath} % disable protrusion for tt fonts
}{}
\makeatletter
\@ifundefined{KOMAClassName}{% if non-KOMA class
  \IfFileExists{parskip.sty}{%
    \usepackage{parskip}
  }{% else
    \setlength{\parindent}{0pt}
    \setlength{\parskip}{6pt plus 2pt minus 1pt}}
}{% if KOMA class
  \KOMAoptions{parskip=half}}
\makeatother
% Make \paragraph and \subparagraph free-standing
\makeatletter
\ifx\paragraph\undefined\else
  \let\oldparagraph\paragraph
  \renewcommand{\paragraph}{
    \@ifstar
      \xxxParagraphStar
      \xxxParagraphNoStar
  }
  \newcommand{\xxxParagraphStar}[1]{\oldparagraph*{#1}\mbox{}}
  \newcommand{\xxxParagraphNoStar}[1]{\oldparagraph{#1}\mbox{}}
\fi
\ifx\subparagraph\undefined\else
  \let\oldsubparagraph\subparagraph
  \renewcommand{\subparagraph}{
    \@ifstar
      \xxxSubParagraphStar
      \xxxSubParagraphNoStar
  }
  \newcommand{\xxxSubParagraphStar}[1]{\oldsubparagraph*{#1}\mbox{}}
  \newcommand{\xxxSubParagraphNoStar}[1]{\oldsubparagraph{#1}\mbox{}}
\fi
\makeatother


\usepackage{longtable,booktabs,array}
\usepackage{calc} % for calculating minipage widths
% Correct order of tables after \paragraph or \subparagraph
\usepackage{etoolbox}
\makeatletter
\patchcmd\longtable{\par}{\if@noskipsec\mbox{}\fi\par}{}{}
\makeatother
% Allow footnotes in longtable head/foot
\IfFileExists{footnotehyper.sty}{\usepackage{footnotehyper}}{\usepackage{footnote}}
\makesavenoteenv{longtable}
\usepackage{graphicx}
\makeatletter
\newsavebox\pandoc@box
\newcommand*\pandocbounded[1]{% scales image to fit in text height/width
  \sbox\pandoc@box{#1}%
  \Gscale@div\@tempa{\textheight}{\dimexpr\ht\pandoc@box+\dp\pandoc@box\relax}%
  \Gscale@div\@tempb{\linewidth}{\wd\pandoc@box}%
  \ifdim\@tempb\p@<\@tempa\p@\let\@tempa\@tempb\fi% select the smaller of both
  \ifdim\@tempa\p@<\p@\scalebox{\@tempa}{\usebox\pandoc@box}%
  \else\usebox{\pandoc@box}%
  \fi%
}
% Set default figure placement to htbp
\def\fps@figure{htbp}
\makeatother


% definitions for citeproc citations
\NewDocumentCommand\citeproctext{}{}
\NewDocumentCommand\citeproc{mm}{%
  \begingroup\def\citeproctext{#2}\cite{#1}\endgroup}
\makeatletter
 % allow citations to break across lines
 \let\@cite@ofmt\@firstofone
 % avoid brackets around text for \cite:
 \def\@biblabel#1{}
 \def\@cite#1#2{{#1\if@tempswa , #2\fi}}
\makeatother
\newlength{\cslhangindent}
\setlength{\cslhangindent}{1.5em}
\newlength{\csllabelwidth}
\setlength{\csllabelwidth}{3em}
\newenvironment{CSLReferences}[2] % #1 hanging-indent, #2 entry-spacing
 {\begin{list}{}{%
  \setlength{\itemindent}{0pt}
  \setlength{\leftmargin}{0pt}
  \setlength{\parsep}{0pt}
  % turn on hanging indent if param 1 is 1
  \ifodd #1
   \setlength{\leftmargin}{\cslhangindent}
   \setlength{\itemindent}{-1\cslhangindent}
  \fi
  % set entry spacing
  \setlength{\itemsep}{#2\baselineskip}}}
 {\end{list}}
\usepackage{calc}
\newcommand{\CSLBlock}[1]{\hfill\break\parbox[t]{\linewidth}{\strut\ignorespaces#1\strut}}
\newcommand{\CSLLeftMargin}[1]{\parbox[t]{\csllabelwidth}{\strut#1\strut}}
\newcommand{\CSLRightInline}[1]{\parbox[t]{\linewidth - \csllabelwidth}{\strut#1\strut}}
\newcommand{\CSLIndent}[1]{\hspace{\cslhangindent}#1}



\setlength{\emergencystretch}{3em} % prevent overfull lines

\providecommand{\tightlist}{%
  \setlength{\itemsep}{0pt}\setlength{\parskip}{0pt}}



 


\usepackage{booktabs}
\usepackage{caption}
\usepackage{longtable}
\usepackage{colortbl}
\usepackage{array}
\usepackage{anyfontsize}
\usepackage{multirow}
\KOMAoption{captions}{tableheading}
\makeatletter
\@ifpackageloaded{caption}{}{\usepackage{caption}}
\AtBeginDocument{%
\ifdefined\contentsname
  \renewcommand*\contentsname{Table of contents}
\else
  \newcommand\contentsname{Table of contents}
\fi
\ifdefined\listfigurename
  \renewcommand*\listfigurename{List of Figures}
\else
  \newcommand\listfigurename{List of Figures}
\fi
\ifdefined\listtablename
  \renewcommand*\listtablename{List of Tables}
\else
  \newcommand\listtablename{List of Tables}
\fi
\ifdefined\figurename
  \renewcommand*\figurename{Figure}
\else
  \newcommand\figurename{Figure}
\fi
\ifdefined\tablename
  \renewcommand*\tablename{Table}
\else
  \newcommand\tablename{Table}
\fi
}
\@ifpackageloaded{float}{}{\usepackage{float}}
\floatstyle{ruled}
\@ifundefined{c@chapter}{\newfloat{codelisting}{h}{lop}}{\newfloat{codelisting}{h}{lop}[chapter]}
\floatname{codelisting}{Listing}
\newcommand*\listoflistings{\listof{codelisting}{List of Listings}}
\makeatother
\makeatletter
\makeatother
\makeatletter
\@ifpackageloaded{caption}{}{\usepackage{caption}}
\@ifpackageloaded{subcaption}{}{\usepackage{subcaption}}
\makeatother
\usepackage{bookmark}
\IfFileExists{xurl.sty}{\usepackage{xurl}}{} % add URL line breaks if available
\urlstyle{same}
\hypersetup{
  pdftitle={Evaluating Retrofit Strategies for Urban Heat Resilience: A Singapore Case Study},
  pdfauthor={Anahita Darvish},
  colorlinks=true,
  linkcolor={blue},
  filecolor={Maroon},
  citecolor={Blue},
  urlcolor={Blue},
  pdfcreator={LaTeX via pandoc}}


\title{Evaluating Retrofit Strategies for Urban Heat Resilience: A
Singapore Case Study}
\author{Anahita Darvish}
\date{2025-12-17}
\begin{document}
\maketitle

\renewcommand*\contentsname{Table of contents}
{
\hypersetup{linkcolor=}
\setcounter{tocdepth}{3}
\tableofcontents
}

\subsection{Introduction}\label{introduction}

The building sector currently comprises a large percentage of worldwide
energy consumption {[}1{]}. With the rise of climate change and energy
efficiency policies, the need to improve energy efficiency in the
existing built environment is vital {[}2{]}. Building energy consumption
is a especially a challenge in the tropics, where high temperatures and
high humidity lead to the use of large amounts of energy for cooling and
de-humdification {[}3{]}. Energy retrofit measures have gained lots of
traction in recent years due to their potential to increase building
energy performance, provide cost savings, and lower environmental
emissions. Facades are a key focus area in retrofit designs as they are
an essential element of the building's performance. The building facade
is the primary interface between the indoor and outdoor environments,
filtering heat, light, and air before reaching the interior {[}4{]}.

Background info on ERM studies then novelty of my work

The objectives of this report are as follows:

\begin{itemize}
\item
  To compare the cooling demand impacts of three passive energy retrofit
  measures
\item
  To examine how retrofit effectiveness varies across multiple blocks in
  the same complex
\item
  To assess whether observed energy savings are robust or block-specific
\end{itemize}

Old stuff: While the design of new, efficient buildings is also a
research area of interest, these projects require immense amounts of
resources and energy. Retrofit measures provide an effective,
sustainable solution to increase building energy performance while
avoiding issues related to land-use and deforestation. Facades are a key
focus in retrofit designs as they are the facilitator between the indoor
and outdoor worlds. Passive-design strategies, in particular, provide
easy to implement and simple interventions that can have a large impact
on the building's total energy consumption.

\begin{itemize}
\tightlist
\item
  Mention the structure of the report, what the reader will find in each
  section etc.
\end{itemize}

\subsection{Methodology}\label{methodology}

\subsubsection{Target buildings and simulation
setup}\label{target-buildings-and-simulation-setup}

A typical high-rise, public residential complex in Singapore was
selected for this study. Everton park consists of seven, 12-storey-high
buildings facing north or south, in the configuration shown in
Figure~\ref{fig-layout}.

\begin{figure}

\centering{

\pandocbounded{\includegraphics[keepaspectratio]{images/everton_layout-01.png}}

}

\caption{\label{fig-layout}Everton park layout}

\end{figure}%

Buildings of interest in this study are the central building, block 7,
and the buildings directly north and south of it, blocks 2 and 5,
respectively. Evaluating multiple blocks allows for a more robust
simulation of energy retrofit measures (ERMs), as they will be applied
to buildings with different relative positions, solar exposure, and
boundary conditions.

Energy consumption in each building was modeled using EnergyPlus, a
widely-used physics based building energy modelling (BEM) program
(source). Energy models developed by {[}5{]} were used as a baseline for
this analysis. All parameters relevant to replicating the models
including floor plans, zoning, construction materials, air-conditioning
settings, and schedules can be referenced in their work. The weather
file used in all simulations is the developed microclimate data by
(source) which proved to perform better than historical or measured
weather datasets for building energy modelling.

Energy consumption due to air-conditioning is the main target area for
this analysis. To represent this as an output meter in EnergyPlus,
Cooling:EnergyTransfer was selected. This output meter represents how
much cooling energy the building needs to maintain comfort. It does this
by measuring how much heat must be removed from all zones to maintain
temperature setpoints. Using this output metric, the percentage change
between the baseline case and the retrofit case was calculated to
identify the significance of the ERM.

\subsubsection{Selection of energy retrofit
measures}\label{selection-of-energy-retrofit-measures}

After conducting a literature review on common ERMs used in energy
simulation models, three ERMs from different disciplines were chosen to
expand the breadth of the study including: design, engineering, and
nature based retrofits. Table~\ref{tbl-erms} lists the ERMs as well as
the building component they correspond to, namely, windows, exterior
walls, and exterior roof. Shading and coating ERMs were simulated
individually by adding and/or adjusting one component in the model. If
the reduction in cooling demand was deemed significant, a combination of
the ERM with a roof ERM was then simulated to determine the maximal
possible energy savings for each block.

\begin{table}

\caption{\label{tbl-erms}Energy retrofit measures for simulation}

\centering{

\fontsize{12.0pt}{14.0pt}\selectfont
\begin{tabular*}{\linewidth}{@{\extracolsep{\fill}}lll}
\toprule
{\bfseries Discipline} & {\bfseries Energy retrofit measures} & {\bfseries Building component} \\ 
\midrule\addlinespace[2.5pt]
Design & Overhang shading & Windows \\ 
Engineering & High reflectance coating & Walls, Roof \\ 
Nature & Roof vegetation & Roof \\ 
\bottomrule
\end{tabular*}

}

\end{table}%

\subsubsection{Baseline case simulation and
verification}\label{baseline-case-simulation-and-verification}

To determine a baseline cooling demand for each of block, the existing
IDF models were simulated with the developed microclimate weather
dataset. The baseline case as well as all retrofit scenarios were
simulated for June 2015. A daily reporting frequency was selected for
the Cooling:EnergyTransfer output meter; therefore, the daily outputs
were summed to calculate the monthly cooling demand. Weekends were not
considered in the analysis and were thus subtracted from the monthly
total. The Cooling:EnergyTransfer output meter returns a thermal energy
(heat removed from the zone). To calculate electricity used for cooling,
Equation~\ref{eq-cop} was applied. Where \(E_{el}\) represents the
electricity input to the AC unit, \(Q_{cool}\) is the heat removed from
the zone for cooling, and COP is the coefficient of performance, which
is set to 3. The simulation period and COP are consistent with {[}6{]}.
\begin{equation}\phantomsection\label{eq-cop}{E_{\mathrm{el}} \;[\mathrm{kWh_{el}}]=\frac{Q_{\mathrm{cool}} \;[\mathrm{kWh_{th}}]}{\mathrm{COP}}}\end{equation}

The models were verified using published data on public housing monthly
electricity consumption {[}7{]} and the percentage of household
electricity used for air-conditioning {[}8{]}. Ranges were selected
around the published values to encompass both conservative and high
electricity use: 250-400 kWh for household monthly electricity
consumption and 20-25\% of electricity use attributable to AC.
Equation~\ref{eq-ac} was used to calculate upper and lower bounds for
the monthly AC electricity consumption of each building. Where
\(p_{AC}\) is the percentage of total monthly electricity consumption
per household used for cooling, \(E_{tot}\) is the total monthly
electricity consumption per household, and \(N_{flats}\)is the total
number of flats in individual residential buildings.

\begin{equation}\phantomsection\label{eq-ac}{
\ E_{AC} = p_{AC} * E_{tot} * N_{flats}
}\end{equation}

Figure~\ref{fig-reference} shows the range of reference values tested
and where each baseline model falls within the range. All baseline
values fell on the more conservative side of the range, which is
plausible because the model uses an ideal loads cooling system. Block 7
shows a very conservative estimation likely because the monthly energy
consumption scaled linearly with the number of flats, but the model does
not, it also considers relative location and shading from surrounding
blocks (idk if I want to claim this already). Since all baseline values
fall within the benchmark ranges, they were deemed suitable for a
comparative analysis of retrofit scenarios.

\begin{figure}

\centering{

\pandocbounded{\includegraphics[keepaspectratio]{design_doc_files/figure-pdf/fig-reference-1.pdf}}

}

\caption{\label{fig-reference}Verification of model}

\end{figure}%

\subsection{Results}\label{results}

\subsubsection{Shading}\label{shading}

Solar shading devices play an important role in building energy and
thermal behavior. In hot and humid climates like Singapore, shading can
prevent excessive solar gain, lowering cooling demand. When implementing
solar shading on windows, it is important to consider both solar
radiation regulation and satisfactory levels of natural light and
visibility to the outdoors {[}9{]}. While many residential buildings
make use of internal shading devices, external devices are more
effective in reducing cooling loads because they can intercept and
reduce solar radiation before it reaches the glass panes {[}10{]}. (Add
mechanism sentence)

For this study, horizontal overhangs on windows were chosen as the
retrofit measure, which corresponds to the EnergyPlus input
Shading:Overhang:Projection. Overhang depths were varied as a fraction
of the window height including values of: 0.25, 0.50, 0.75, and 1.00.
Windows in this model are only on the north and south facing facades. To
test differences in response of the north and south facing facades,
overhangs were applied on each facade individually then on both. Since
overhangs were not a part of the baseline model, a placeholder overhang
was created on each window before running a parametric simulation for
overhang depths.

Figure~\ref{fig-shading} shows the results for the overhang shading
parametric analysis for north facing windows, south facing windows, and
all windows for each block. The results indicate that as the overhang
depth increases, the percent change in cooling demand also increases.
However, the results are of such a small magnitude, \(\pm\) 0.1\%
change, that they are deemed insignificant and will not be used in
comparative analysis with other retrofits.

Shading likely did not have an effect on the overall cooling energy
demand due to the climate sensitive design of the buildings. The
buildings all have a north-south orientation with no windows on the east
and west facing facades, strategies to minimize solar heat gains in
Singapore {[}3{]}. Since the sun is almost directly overhead during June
in Singapore, windows on the north and south facades get very little sun
exposure, thus the shading had a limited effect. Also, the baseline
models consider both corridor and inter-building shading which may
already provide sufficient shading on the north and south facades
(should I include this line?).

\begin{figure}

\centering{

\pandocbounded{\includegraphics[keepaspectratio]{design_doc_files/figure-pdf/fig-shading-1.pdf}}

}

\caption{\label{fig-shading}Effects of overhang shading on windows}

\end{figure}%

\subsubsection{Coating}\label{coating}

Reflective or cool materials are considered an effective method for
decreasing building thermal loads (source coating). Light color
coatings, for example, can be applied to any building surface to
increase albedo, the fraction of sunlight reflected by a surface. (Add
mechanism sentence)

To implement cool coatings in the models, the solar absorptance of the
exterior walls and roof materials were varied from a range of dark to
cool, including values: 0.7, 0.5, 0.4, 0.3, and 0.2. The coatings were
applied systemiatically, to identify if any facade wall had a larger
response. The general methodology in EnergyPlus was as follows: a new
material was created with the desired solar absorptance, followed by a
new construction of the exterior wall and/or roof, and finally, the new
construction was applied to the desired facade.

Figure~\ref{fig-coolwall} shows the results for parametric analysis of
cool coatings on each facade wall. The results indicate that as solar
absorptance decreases, the percent change in cooling demand increases.
Thus, for the remaining retrofit scenarios, a solar absorptance of 0.2
will be used. Additionally, the facade walls all demonstrate a similar,
limited, response with less than a 1\% change in cooling demand.
Indicating that no facade wall, alone, yields a more beneficial
response.

\begin{figure}

\centering{

\pandocbounded{\includegraphics[keepaspectratio]{design_doc_files/figure-pdf/fig-coolwall-1.pdf}}

}

\caption{\label{fig-coolwall}Cool coating on walls}

\end{figure}%

Next, the application of coating on all facade walls was compared to
coating on the roof, displayed in Figure~\ref{fig-coolroof}.

\begin{figure}

\centering{

\pandocbounded{\includegraphics[keepaspectratio]{design_doc_files/figure-pdf/fig-coolroof-1.pdf}}

}

\caption{\label{fig-coolroof}Cool coating on roof}

\end{figure}%

The data shows that coatings on the roof have a much larger impact than
coatings on the exterior walls. Best case scenarios for rooptop and wall
savings are 8.75\% and 2.06\%, respectively. This is as expected, since
the roof is far more sun exposed due to Singapore's sun path. It is
interesting to note that block 7, the central building, outperformed
both surrounding buildings, which had similar cooling demand savings.
The cool coating on block 7 more than doubled the savings of blocks 2
and 5.

\subsubsection{Green roof}\label{green-roof}

Green roofs (GRs) are a form of urban greenery that is beneficial in
reducing a building's heat gain and energy consumption (and mitigating
the Urban Heat Island effect) {[}3{]}. They can help provide thermal
insulation to the interior spaces below it, maximizing indoor thermal
comfort and minimizing cooling needs {[}11{]}. The benefits of GRs can
be explained through four main mechanisms including: insulation,
evapotranspiration, shading, and wind barrier effect {[}12{]}.

EnergyPlus utilizes a built-in module to simulate GRs, based on a heat
balance principle between the soil layer and vegetation layer {[}13{]}.
The module accounts for sensible heat flux, latent heat flux, and long-
and short-wave radiation.

To add a GR to the baseline model, a new vegetation material was
created, followed by a new construction. The new construction was then
applied to the exterior roof, following a similar setup to the cool
coatings. Two cases of GRs were simulated with details presented in
Table~\ref{tbl-details}. For this study, only the leaf area index and
plant height were varied between cases. All other parameters related to
the vegetation material, such as substrate thickness and soil
conductivity, were set to the default EnergyPlus values. The two cases
represent a green roof with grass, which covers the roof more
extensively (higher LAI) but has a shorter plant height, and shrubs,
which cover less of the roof area (lower LAI) and have a higher plant
height.

\begin{table}

\caption{\label{tbl-details}Details of green roof}

\centering{

\fontsize{12.0pt}{14.0pt}\selectfont
\begin{tabular*}{\linewidth}{@{\extracolsep{\fill}}llll}
\toprule
{\bfseries Parameter} & {\bfseries Grass} & {\bfseries Shrubs} & {\bfseries Source} \\ 
\midrule\addlinespace[2.5pt]
Height of plants (m) & 0.15 & 1.00 & Parametric variable \\ 
Leaf area index (LAI) & 3.00 & 1.50 & Parametric variable \\ 
Leaf reflectivity & 0.20 & 0.20 & Default \\ 
Leaf emissivity & 0.95 & 0.95 & Default \\ 
Minimum stomatal resistance (s/m) & 180.00 & 180.00 & Default \\ 
Substrate thickness (m) & 0.10 & 0.10 & Default \\ 
Conductivity of dry soil (W/m\textperiodcenteredK) & 0.35 & 0.35 & Default \\ 
Thermal absorptance & 0.90 & 0.90 & Default \\ 
\bottomrule
\end{tabular*}

}

\end{table}%

Each green roof was paired with a cool coating on all facade exterior
walls for simulation. Figure~\ref{fig-comparison} shows a comparison
between both GRs and complete coating, on all facade exteriors. The GRs
on all blocks performed similarly, with grass demonstrating slightly
better cooling demand savings. For blocks 2 and 5, the cool coating
slightly outperformed the GRs. For block 7, however, the cool coating
more than doubled the energy savings of the GRs, which is consistent
with the results seen when comparing the coating on the walls vs the
roof.

\begin{figure}

\centering{

\pandocbounded{\includegraphics[keepaspectratio]{design_doc_files/figure-pdf/fig-comparison-1.pdf}}

}

\caption{\label{fig-comparison}Comparision between green roof with wall
coatings and complete coating}

\end{figure}%

\subsubsection{Evaluation}\label{evaluation}

Each ERM tested, with the corresponding block, is categorized in
Table~\ref{tbl-eval} based on their energy rating. The energy rating is
a function of \(p\), the percentage reduction in cooling energy demand
which is calculated using Equation~\ref{eq-percentage}.

\begin{equation}\phantomsection\label{eq-percentage}{
p = \frac{E_{retrofit} - E_{baseline}}{E_{baseline}} * 100
}\end{equation}

\begin{table}

\caption{\label{tbl-eval}Energy retrofit measures evaluation}

\centering{

\fontsize{12.0pt}{14.0pt}\selectfont
\begin{tabular*}{\linewidth}{@{\extracolsep{\fill}}lll}
\toprule
Monthly Energy Saving (\%) & {\bfseries Energy Rating} & {\bfseries ERMs (block)} \\ 
\midrule\addlinespace[2.5pt]
5 < p \ensuremath{\leq} 10 & High & CR (7), CR+CW (7) \\ 
2 < p \ensuremath{\leq} 5 & Moderate & CR (2,5), CR+CW (2,5), GR+CW (2,7,5) \\ 
p \ensuremath{\leq} 2 & Low & CW (2,7,5), GR (2,7,5), S (2,7,5) \\ 
\bottomrule
\end{tabular*}

}

\end{table}%

\subsection{Conclusion}\label{conclusion}

bla bla bla

\subsection*{References}\label{references}
\addcontentsline{toc}{subsection}{References}

\phantomsection\label{refs}
\begin{CSLReferences}{0}{0}
\bibitem[\citeproctext]{ref-SARIHI2021102525}
\CSLLeftMargin{{[}1{]} }%
\CSLRightInline{S. Sarihi, F. Mehdizadeh Saradj, and M. Faizi, {``A
critical review of façade retrofit measures for minimizing heating and
cooling demand in existing buildings,''} \emph{Sustainable Cities and
Society}, vol. 64, p. 102525, 2021, doi:
\url{https://doi.org/10.1016/j.scs.2020.102525}.}

\bibitem[\citeproctext]{ref-martinezFundamentalsFacadeRetrofit2015}
\CSLLeftMargin{{[}2{]} }%
\CSLRightInline{A. Martinez, M. Patterson, A. Carlson, and D. Noble,
{``Fundamentals in façade retrofit practice,''} \emph{Defining the
future of sustainability and resilience in design, engineering and
construction}, vol. 118, pp. 934--941, Jan. 2015, doi:
\href{https://doi.org/10.1016/j.proeng.2015.08.534}{10.1016/j.proeng.2015.08.534}.}

\bibitem[\citeproctext]{ref-zotero-item-221}
\CSLLeftMargin{{[}3{]} }%
\CSLRightInline{\emph{Building planning and massing}. in Green building
platinum. Singapore: {Building and Construction Authority}, 2010.}

\bibitem[\citeproctext]{ref-richards2014importance}
\CSLLeftMargin{{[}4{]} }%
\CSLRightInline{D. Richards, {``The importance of fa{ç}ade design,''}
\emph{Sustainable Retrofitting of Commercial Buildings: Cool Climates},
p. 140, 2014.}

\bibitem[\citeproctext]{ref-XU2022103775a}
\CSLLeftMargin{{[}5{]} }%
\CSLRightInline{L. Xu \emph{et al.}, {``Better understanding on impact
of microclimate information on building energy modelling performance for
urban resilience,''} \emph{Sustainable Cities and Society}, vol. 80, p.
103775, 2022, doi: \url{https://doi.org/10.1016/j.scs.2022.103775}.}

\bibitem[\citeproctext]{ref-XU2022103775b}
\CSLLeftMargin{{[}6{]} }%
\CSLRightInline{L. Xu \emph{et al.}, {``Better understanding on impact
of microclimate information on building energy modelling performance for
urban resilience,''} \emph{Sustainable Cities and Society}, vol. 80, p.
103775, 2022, doi: \url{https://doi.org/10.1016/j.scs.2022.103775}.}

\bibitem[\citeproctext]{ref-zotero-item-225}
\CSLLeftMargin{{[}7{]} }%
\CSLRightInline{{``Singapore {Energy Statistics} 2025,''} \emph{Energy
market authority}.
https://www.ema.gov.sg/resources/singapore-energy-statistics.}

\bibitem[\citeproctext]{ref-zotero-item-222}
\CSLLeftMargin{{[}8{]} }%
\CSLRightInline{{``Four {In Five Households Motivated To Save Energy If
They Can Save Money}: {NEA Study},''} \emph{National Environment
Agency}, May 2018.}

\bibitem[\citeproctext]{ref-su6085354}
\CSLLeftMargin{{[}9{]} }%
\CSLRightInline{C. Carletti, F. Sciurpi, and L. Pierangioli, {``The
energy upgrading of existing buildings: {Window} and shading device
typologies for energy efficiency refurbishment,''}
\emph{Sustainability}, vol. 6, no. 8, pp. 5354--5377, 2014, doi:
\href{https://doi.org/10.3390/su6085354}{10.3390/su6085354}.}

\bibitem[\citeproctext]{ref-KIM2012105}
\CSLLeftMargin{{[}10{]} }%
\CSLRightInline{G. Kim, H. S. Lim, T. S. Lim, L. Schaefer, and J. T.
Kim, {``Comparative advantage of an exterior shading device in thermal
performance for residential buildings,''} \emph{Energy and Buildings},
vol. 46, pp. 105--111, 2012, doi:
\href{https://doi.org/10.1016/j.enbuild.2011.10.040}{10.1016/j.enbuild.2011.10.040}.}

\bibitem[\citeproctext]{ref-ZHAO2023113668}
\CSLLeftMargin{{[}11{]} }%
\CSLRightInline{Y. Zhao \emph{et al.}, {``Beating urban heat:
{Multimeasure-centric} solution sets and a complementary framework for
decision-making,''} \emph{Renewable and Sustainable Energy Reviews},
vol. 186, p. 113668, 2023, doi:
\href{https://doi.org/10.1016/j.rser.2023.113668}{10.1016/j.rser.2023.113668}.}

\bibitem[\citeproctext]{ref-dahanayakeComparingReductionBuilding2018}
\CSLLeftMargin{{[}12{]} }%
\CSLRightInline{K. C. Dahanayake and C. L. Chow, {``Comparing reduction
of building cooling load through green roofs and green walls by
{EnergyPlus} simulations,''} \emph{Building Simulation}, vol. 11, no. 3,
pp. 421--434, Jun. 2018, doi:
\href{https://doi.org/10.1007/s12273-017-0415-7}{10.1007/s12273-017-0415-7}.}

\bibitem[\citeproctext]{ref-SAILOR20081466}
\CSLLeftMargin{{[}13{]} }%
\CSLRightInline{D. J. Sailor, {``A green roof model for building energy
simulation programs,''} \emph{Energy and Buildings}, vol. 40, no. 8, pp.
1466--1478, 2008, doi:
\href{https://doi.org/10.1016/j.enbuild.2008.02.001}{10.1016/j.enbuild.2008.02.001}.}

\end{CSLReferences}




\end{document}
